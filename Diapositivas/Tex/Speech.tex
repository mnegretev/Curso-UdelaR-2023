\section{Herramientas para HRI}

\begin{frame}\frametitle{Síntesis de voz con SoundPlay}
  \begin{itemize}
  \item Es un paquete que permite reproducir archivos \texttt{.wav} o \texttt{.ogg}, sonidos predeterminados y síntesis de voz.
  \item La síntesis de voz se hace utilizando Festival (\url{http://www.cstr.ed.ac.uk/projects/festival/}).
  \item Para sintetizar voz, basta con correr el nodo \texttt{soundplay\_node} y publicar un mensaje de tipo \texttt{sound\_play/SoundRequest} con lo siguiente:
    \begin{itemize}
    \item msg\_speech.sound   = -3                 
    \item msg\_speech.command = 1                  
    \item msg\_speech.volume  = 1.0                
    \item msg\_speech.arg2    = ``voz a utilizar''
    \item msg\_speech.arg = ``texto a sintetizar''
    \end{itemize}
  \end{itemize}
\end{frame}

\begin{frame}[containsverbatim]\frametitle{Ejercicio 12 - Síntesis de voz}
  Ejecute el comando:
  \begin{lstlisting}
    roslaunch bring_up speech_synthesis.launch
  \end{lstlisting}
  En otra terminal, ejecute el comando:
  \begin{lstlisting}
    rosrun students assignment12.py "my first synthetized voice"
  \end{lstlisting}
  Para instalar más voces:
  \begin{itemize}
  \item Ejecute el comando \texttt{sudo apt-get install festvox-<voz deseada>}
  \item Para ver qué voces se tienen instaladas: \texttt{ls /usr/share/festival/voices/english/}
  \end{itemize}
\end{frame}

\begin{frame}[containsverbatim]\frametitle{Ejercicio 12 - Síntesis de voz}
  Modifique el archivo \texttt{catkin\_ws/src/students/scripts/assignment12.py} y cambie la voz a utilizar en el mensaje SoundRequest.
  \lstinputlisting[language=Python,firstnumber=17]{Codes/SpeechSynthesis.py}
  El nombre de la voz se compone de \texttt{voice\_} más el nombre que aparece en la carpeta \texttt{/usr/share/festival/voices/english/}. 
\end{frame}

\begin{frame}\frametitle{Reconocimiento de voz con Pocketsphinx}
  Pocketsphinx es un \textit{toolkit} open source desarrollado por la Universidad de Carnegie Mellon (\url{https://cmusphinx.github.io/}).
  \begin{itemize}
  \item Aunque el toolbox original no está hecho específicamente para ROS, ya existen varios repositorios con nodos ya implementados que integran ROS y Pocketsphinx:
    \begin{itemize}
    \item \url{https://github.com/mikeferguson/pocketsphinx}
    \item \url{https://github.com/Pankaj-Baranwal/pocketsphinx}
    \end{itemize}
  \item El usuario debe estar agregado al grupo \textit{audio} para el correcto funcionamiento: \texttt{sudo usermod -a -G audio <user\_name>}
  \end{itemize}
  \begin{itemize}
  \item Se puede hacer reconocimiento usando una lista de palabras, un modelo de lenguaje o una gramática.
  \item Se utilizarán gramáticas y sus correspondientes diccionarios.
  \item Para construir diccionarios, visitar \url{https://cmusphinx.github.io/wiki/tutorialdict/}
  \item Para construir gramáticas, visitar \url{https://www.w3.org/TR/2000/NOTE-jsgf-20000605/}
  \end{itemize}
\end{frame}

\begin{frame}\frametitle{Ejercicio 13 - Reconocimiento de voz}
  \begin{enumerate}
  \item Verifique el volumne del micrófono
  \item Inspeccione el archivo \texttt{catkin\_ws/src/hri/sprec\_pocketsphinx/vocab/final\_project.gram} para ver las frases que se pueden reconocer de acuerdo con la gramática.
  \item Ejecute el comando \texttt{roslaunch bring\_up speech\_recognition.launch}
  \item En otra terminal, ejecute el comanto \texttt{rostopic echo /hri/sp\_rec/recognized}
  \item Pruebe el reconocimiento de voz con alguna de las siguientes frases:
    \begin{enumerate}
    \item \texttt{Robot, take the pringles to the table}
    \item \texttt{Robot, take the drink to the table}
    \item \texttt{Robot, take the pringles to the kitchen}
    \item \texttt{Robot, take the drink to the kitchen}
    \end{enumerate}
  \end{enumerate}
\end{frame}


\section{Proyecto final}

\begin{frame}\frametitle{Proyecto final: robot de servicio}
\end{frame}
