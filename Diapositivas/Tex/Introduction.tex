\section{Introducción}
\begin{frame}\frametitle{Presentación del curso}
  \textbf{Objetivos:}
  \begin{itemize}
  \item Aprender los conceptos básicos para operar un robot móvil autónomo
  \item Implementar dichos conceptos en un ambiente simulado
  \item Familiarizar al estudiante con la plataforma ROS
  \end{itemize}
\end{frame}

\begin{frame}\frametitle{Contenido}
  \begin{enumerate}
  \item Introducción y generalidades
    \begin{itemize}
    \item Componentes básicos de un robot móvil
    \item Herramientas de software para el desarrollo de robots móviles
    \end{itemize}
  \item Planeación de movimientos
    \begin{itemize}
    \item El problema de la planeación de movimientos
    \item Mapas geométricos y topológicos
    \item Celdas de ocupación y diagramas de Voronoi
    \item Planeación de rutas mediante búsqueda en grafos
    \item Modelos cinemáticos: diferencial, omnidireccional, Ackermann
    \item Control de posición y seguimiento de trayectorias
    \item Campos potenciales artificiales
    \end{itemize}
  \item Mapeo y localización
    \begin{itemize}
    \item Localización mediante filtro de Kalman extendido
    \item Localización mediante filtros de partículas
    \item Creación de mapas mediante agrupamiento
    \item Localización y mapeo simultáneos
    \end{itemize}
  \end{enumerate}
\end{frame}

\begin{frame}\frametitle{Contenido}
  \begin{enumerate}
    \setcounter{enumi}{3}
  \item Conceptos básicos de visión artificial
    \begin{itemize}
    \item Imágenes y espacios de color
    \item Operadores morfológicos
    \item Extracción de características geométricas
    \item Reconocimiento mediante redes neuronales artificiales
    \item Nubes de puntos
    \end{itemize}
  \item Conceptos básicos de manipulación
    \begin{itemize}
    \item Movimiento de cuerpo rígido
    \item Cinemática directa
    \item Cinemática inversa por métodos numéricos
    \item Planeación y seguimiento de trayectorias
    \end{itemize}
  \item Herramientas para la interacción humano-robot
    \begin{itemize}
    \item Síntesis de voz con la biblioteca Festival
    \item Reconocimiento de voz con la biblioteca CMU Sphinx
    \item Reconocimiento de gestos con la biblioteca OpenPose
    \end{itemize}
  \end{enumerate}
\end{frame}

\begin{frame}\frametitle{Contenido}
  \textbf{Bibliografía recomendada:}
  \begin{itemize}
    \item \url{https://drive.google.com/drive/folders/1gb7VQJG5eUkCvCginRHHGn5lez6VASBJ?usp=sharing}
    \item \url{https://drive.google.com/drive/folders/1Epl2b51xEJzCvzfugBD1i7xGdKYdJucy?usp=sharing}
  \end{itemize}
  
\end{frame}

