\section{Herramientas para HRI}

\begin{frame}\frametitle{Síntesis de voz con SoundPlay}
  \begin{itemize}
  \item Es un paquete que permite reproducir archivos \texttt{.wav} o \texttt{.ogg}, sonidos predeterminados y síntesis de voz.
  \item La síntesis de voz se hace utilizando Festival (\url{http://www.cstr.ed.ac.uk/projects/festival/}).
  \item Para sintetizar voz, basta con correr el nodo \texttt{soundplay\_node} y publicar un mensaje de tipo \texttt{sound\_play/SoundRequest} con lo siguiente:
    \begin{itemize}
    \item msg\_speech.sound   = -3                 
    \item msg\_speech.command = 1                  
    \item msg\_speech.volume  = 1.0                
    \item msg\_speech.arg2    = ``voz a utilizar''
    \item msg\_speech.arg = ``texto a sintetizar''
    \end{itemize}
  \end{itemize}
\end{frame}

\begin{frame}[containsverbatim]\frametitle{Ejercicio 12 - Síntesis de voz}
  Ejecute el comando:
  \begin{lstlisting}
    roslaunch bring_up speech_synthesis.launch
  \end{lstlisting}
  En otra terminal, ejecute el comando:
  \begin{lstlisting}
    rosrun students assignment12.py "my first synthetized voice"
  \end{lstlisting}
  Para instalar más voces:
  \begin{itemize}
  \item Ejecute el comando \texttt{sudo apt-get install festvox-<voz deseada>}
  \item Para ver qué voces se tienen instaladas: \texttt{ls /usr/share/festival/voices/english/}
  \end{itemize}
\end{frame}

\begin{frame}[containsverbatim]\frametitle{Ejercicio 12 - Síntesis de voz}
  Modifique el archivo \texttt{catkin\_ws/src/students/scripts/assignment12.py} y cambie la voz a utilizar en el mensaje SoundRequest.
  \lstinputlisting[language=Python,firstnumber=17]{Codes/SpeechSynthesis.py}
  El nombre de la voz se compone de \texttt{voice\_} más el nombre que aparece en la carpeta \texttt{/usr/share/festival/voices/english/}. 
\end{frame}

\section{Proyecto final}

\begin{frame}\frametitle{Proyecto final: robot de servicio}
\end{frame}
